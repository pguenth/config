% "Builtin normal mode keys of Vifm" cheatsheet
%
% Copyright (C) 2020 xaizek.
% Made available under the terms of Apache 2.0 license.
%
% Note that LaTeX knowledge of the author is extermely limited and the code
% below is probably of quite low quality.

\documentclass[landscape]{article}

\usepackage[latin1]{inputenc}
\usepackage{tikz}

\usepackage{color}
\usepackage{colortbl}
\usepackage{fontawesome}
\usepackage{setspace}
\usepackage{tcolorbox}
\usepackage{multicol}

\usepackage[paperwidth=32.5cm, paperheight=22cm] {geometry}
\geometry{top=2mm,bottom=0mm,left=-3mm,right=0mm}

\usepackage[oldsyntax]{stackengine}
\newcommand\makeblock[3]{\parbox[t]{#1}{\textbf{#2}\\[2pt]#3}}
\Sstackgap=1.5ex

\newcommand\legendbox[1]{\tcbox[colframe=#1,colback=#1,colupper=#1,nobeforeafter,size=small,box align=base]{\rule{1ex}{1ex}}}

\setlength{\tabcolsep}{0.2em}

% disable page numbers
\pagestyle{empty}

\newlength\unit
\newlength\sep
\newlength\base

\setlength{\unit}{0.3cm}
\setlength{\sep}{0.4\unit} %
\setlength{\base}{4\unit}

% comment this line for big keyboard (and increase page size)
\def\shortkeyboard{\true}

\definecolor{util}{gray}{0.5}
\definecolor{note}{RGB}{75,75,205}
\definecolor{keynameRed}{RGB}{225,75,75}
\definecolor{keyname}{RGB}{50,50,50}

% tango colors lvl 1
% \definecolor{colorModifier}{RGB}{239,  41,  41}
%
% \definecolor{groupWS}{RGB}{252, 233,  79}
% \definecolor{groupScreens}{RGB}{138, 226,  52}
% \definecolor{groupLayout}{RGB}{252, 175,  62}
% \definecolor{groupMoveWindows}{RGB}{114, 159, 207}
% \definecolor{groupApplications}{RGB}{173, 127, 168}
% \definecolor{groupSwitchWindows}{RGB}{233, 185, 110}

\definecolor{colorModifier}{RGB}{50,50,50}
\definecolor{colorModifierText}{RGB}{240,240,240}

\definecolor{groupWS}{RGB}{212, 118, 110}
\definecolor{groupScreens}{RGB}{212, 152, 110}
\definecolor{groupLayout}{RGB}{201, 212, 110}
\definecolor{groupMoveWindows}{RGB}{156, 212, 110}
\definecolor{groupApplications}{RGB}{111, 147, 190}
\definecolor{groupSwitchWindows}{RGB}{162, 123, 176}

% \definecolor{groupWSKey}{
% \definecolor{groupScreensKey}{RGB}{212, 152, 110}
% \definecolor{groupLayoutKey}{RGB}{201, 212, 110}
% \definecolor{groupMoveWindowsKey}{RGB}{156, 212, 110}
% \definecolor{groupApplicationsKey}{RGB}{72, 119, 175}
% \definecolor{groupSwitchWindowsKey}{RGB}{162, 123, 176}
 
\newcommand{\s}[1]{{\color{util}#1}}

\newcommand{\note}[1]{{\color{note}(#1)}}

\newcommand{\shortcut}[1]{\texttt{#1}}

\newcommand{\keyname}[1]{{\color{keyname}\textbf{\footnotesize#1}}}
\newcommand{\keynameRed}[1]{{\color{keynameRed}\textbf{\footnotesize#1}}}
\newcommand{\keydesc}[1]{
    \begin{spacing}{0.75}
        \sffamily\scriptsize#1
    \end{spacing}
}

\tikzset{bicolor/.style args={#1 and #2}{
  path picture={
    \tikzset{rounded corners=0}
    \fill [#1] (path picture bounding box.west)
      rectangle (path picture bounding box.north east);
    \fill [#2] (path picture bounding box.west)
      rectangle (path picture bounding box.south east);
}}}

\def\fkeyd#1#2#3#4 {
    \draw[fill=#4] ++(\X,\Y)       rectangle +(\base,\base)
           +(0,0.85\base) node[text width=10cm,text height=1,anchor=north west] {\keyname{#1}\vspace{3px}\keydesc{#2}};
    \setlength{\X}{\X + \base + #3\sep}
}
\def\unusedfkey#1#2 { % the second argument is just to allow spaces before the first one
    \fill[very nearly transparent,gray]
         ++(\X,\Y) rectangle +(\base,\base);
    \draw[nearly transparent,gray]
         +(\X,\Y+0.85*\base) node[text width=10cm,text height=1,anchor=north west] {\keyname{#1}};
    \setlength{\X}{\X + \base + #2\sep}
}

\def\longfkeyd(#1)#2#3#4 {
    \draw[fill=#4] ++(\X,\Y)       rectangle +(#1,1\base)
           +(0,0.65\base) node[text width=10cm,text height=1,text ragged,anchor=north west] {\keyname{#2}\vspace{3px}\keydesc{#3}};
    \setlength{\X}{\X + #1 + \sep}
}
\def\longfkeydmod(#1)#2#3 {
    \draw[fill=colorModifier] ++(\X,\Y)       rectangle +(#1,1\base)
           +(0,0.8\base) node[color=colorModifierText, text width=10cm,text height=1,text ragged,anchor=north west] {\keynameRed{#2}\vspace{3px}\keydesc{#3}};
    \setlength{\X}{\X + #1 + \sep}
}
\def\longfkeydd(#1)#2#3#4,#5#6 {
    \path[bicolor={#5 and #6},draw=none] (\X,\Y) rectangle ++(#1, \base);
    \draw ++(\X,\Y)       rectangle +(#1,1\base)
           +(0,0.89\base) node[text width=10cm,text height=1,text ragged,anchor=north west] {\keyname{Shift + #2}\vspace{1px}\keydesc{#3}}
           +(0,0.42\base) node[text width=10cm,text height=1,text ragged,anchor=north west] {\keyname{#2}\vspace{1px}\keydesc{#4}};
    \draw [densely dotted] ++(\X,\Y+0.5\base) -- +(#1,0);
    \setlength{\X}{\X + #1 + \sep}
}
\def\unusedlongfkey(#1)#2#3 { % the third argument is just to allow spaces before the second one
    \fill[very nearly transparent,gray]
         ++(\X,\Y) rectangle +(#1,\base);
    \draw[nearly transparent,gray]
         +(\X+0.1\base,\Y+0.5\base) node[text width=2,text height=1] {\keyname{#2}};
    \setlength{\X}{\X + #1 + \sep}
}

\def\keyd#1#2,#3#4,#5#6 {
    \path[bicolor={#5 and #6}, draw=none] (\X,\Y) rectangle ++(\base, \base);
    \draw ++(\X,\Y) rectangle +(\base,\base)
           +(0.47\base,0.8\base) node[text width=1cm,text height=1,anchor=north east,align=center] {\keyname{#1}}
           +(0.47\base,0.3\base) node[text width=1cm,text height=1,anchor=north east,align=center] {\keyname{#3}}
           +(0.23\base,0.72\base) node[text width=1.4cm,text height=1,text ragged,anchor=west,align=left] {\keydesc{#2}}
           +(0.23\base,0.22\base) node[text width=1.4cm,text height=1,text ragged,anchor=west,align=left] {\keydesc{#4}};

    \draw [densely dotted] ++(\X,\Y+0.5\base) -- +(\base,0);
    \setlength{\X}{\X + \base + \sep}
}
\def\keydd#1#2,#3#4,#5#6 {
    \draw ++(\X,\Y) rectangle +(\base,\base)
           +(0.47\base,0.88\base)   node[text width=1cm,text height=1,anchor=north east,align=center] {\keyname{#1}}
           +(0.47\base,0.53\base)       node[text width=1cm,text height=1,anchor=north east,align=center] {\keyname{#3}}
           +(0.47\base,0.2\base)       node[text width=1cm,text height=1,anchor=north east,align=center] {\keyname{#5}}
           +(0.23\base,0.8\base) node[text width=1.4cm,text height=1,text ragged,anchor=west] {\keydesc{#2}}
           +(0.23\base,0.47\base)         node[text width=1.4cm,text height=1,text ragged,anchor=west] {\keydesc{#4}}
           +(0.23\base,0.13\base)         node[text width=1.4cm,text height=1,text ragged,anchor=west] {\keydesc{#6}};

    \draw [densely dotted] ++(\X,\Y+0.67\base) -- +(\base,0);
    \draw [densely dotted] ++(\X,\Y+0.33\base) -- +(\base,0);
    \setlength{\X}{\X + \base + \sep}
}

\def\unusedkeyd#1#2,#3#4 {
    \fill[very nearly transparent,gray]
         ++(\X,\Y) rectangle +(\base,\base);
    \draw[nearly transparent,gray]
           +(\X+0.47\base,\Y+0.8\base) node[text width=1cm,text height=1,anchor=north east,align=center] {\keyname{#1}}
           +(\X+0.47\base,\Y+0.3\base) node[text width=1cm,text height=1,anchor=north east,align=center] {\keyname{#3}}
           +(\X+0.23\base,\Y+0.72\base) node[text width=1.4cm,text height=1,text ragged,anchor=west,align=left] {\keydesc{#2}}
           +(\X+0.23\base,\Y+0.22\base) node[text width=1.4cm,text height=1,text ragged,anchor=west,align=left] {\keydesc{#4}};

    \draw[nearly transparent,densely dotted]
         ++(\X,\Y+0.5\base) -- +(\base,0);
    \setlength{\X}{\X + \base + \sep}
}

\def\unusedkeydHalf#1#2,#3#4 {
    \fill[very nearly transparent,gray]
         ++(\X,\Y) rectangle +(\base,\base);
    \draw[nearly transparent,gray]
           +(\X+0.47\base,\Y+0.8\base) node[text width=1cm,text height=1,anchor=north east,align=center] {\keyname{#1}}
           +(\X+0.23\base,\Y+0.72\base) node[text width=1.4cm,text height=1,text ragged,anchor=west,align=left] {\keydesc{#2}};
    \draw
           +(\X+0.47\base,\Y+0.3\base) node[text width=1cm,text height=1,anchor=north east,align=center] {\keyname{#3}}
           +(\X+0.23\base,\Y+0.22\base) node[text width=1.4cm,text height=1,text ragged,anchor=west,align=left] {\keydesc{#4}};

    \draw[densely dotted]
         ++(\X,\Y+0.5\base) -- +(\base,0);
    \setlength{\X}{\X + \base + \sep}
}

\def\unusedkey#1,#2 {
    \fill[very nearly transparent,gray]
         ++(\X,\Y) rectangle +(\base,\base);
    \draw[nearly transparent,gray]
         (\X+0.1\base,\Y+0.6\base) node[text width=1,text height=1,anchor=south] {\keyname{#1}}
         +(0.0,-0.5\base)          node[text width=1,text height=1,anchor=south] {\keyname{#2}};

    \draw[nearly transparent,densely dotted]
         ++(\X,\Y+0.5\base) -- +(\base,0);
    \setlength{\X}{\X + \base + \sep}
}
\def\unusedlongkey(#1)#2,#3 {
    \fill[very nearly transparent,gray]
         ++(\X,\Y) rectangle +(#1,\base);
    \draw[nearly transparent,gray]
          (\X+0.1\base,\Y+0.5\base) node[text width=2,text height=1,anchor=south] {\keyname{#2}}
         +(0.0,-0.5\base)           node[text width=2,text height=1,anchor=south] {\keyname{#3}};

    \draw[nearly transparent,densely dotted] ++(\X,\Y+0.5\base) -- +(#1,0);
    \setlength{\X}{\X + #1 + \sep}
}

\begin{document}

\begin{tikzpicture}[scale=1.6]
    \tikzstyle{every path}=[draw]

    \pgfsetcornersarced{\pgfpoint{1mm}{1mm}}

    \newlength\X
    \newlength\Y

    \Y=0\unit

    \draw (0\unit,\Y + 0.9\unit)
          node[text width=\textwidth,text height=10,text centered,anchor=west]
          {\LARGE \scshape XMonad keybindings (haug)};
    
    \setlength{\Y}{\Y - \base - \unit + \sep}
    % F row
    \newcommand{\frowsep}{6}
    \X=0\unit
    \pgfmathparse{2*\frowsep}
    \keyd          {Esc}{ranger},{Esc}{terminal},{groupApplications}{groupApplications}
    \unusedfkey           {F1}
    \unusedfkey      {F2}
    \unusedfkey           {F3}
    \fkeyd           {F4}{kill current\\window}{\frowsep}{groupApplications}
    
    \unusedfkey           {F5}
    \unusedfkey           {F6}
    \unusedfkey           {F7}
    \unusedfkey           {F8}{\frowsep}
    \unusedfkey      {F9}
    \unusedfkey           {F10}
    \unusedfkey      {F11}
    \unusedfkey      {F12}
\ifx \shortkeyboard \defined
    \setlength{\X}{\X + 0.5\base - \sep}
    \unusedfkey     {Print\\Screen}
    \unusedfkey          {Scroll\\Lock}
    \unusedfkey          {Pause}
\fi
    
    \setlength{\Y}{\Y - \base - \unit - \sep}
    % numeric row
    \X=0\unit
    \keyd           {\texttildelow}{move window to previous screen},{`}{move window to next screen},{groupMoveWindows}{groupMoveWindows}
    \keyd           !{move window to WS1},1{go to WS1},{groupMoveWindows}{groupWS}
    \keyd           @{move window to WS2},2{go to WS2},{groupMoveWindows}{groupWS}
    \keyd           \#{move window to WS3} ,3{go to WS3},{groupMoveWindows}{groupWS}
    \keyd           {\textdollar}{move window to WS4},4{go to WS4},{groupMoveWindows}{groupWS}
    \keyd     \%{move window to WS5},5{go to WS5},{groupMoveWindows}{groupWS}
    \keyd           \^{move window to WS6},6{go to WS6},{groupMoveWindows}{groupWS}
    \keyd           \&{move window to WS7},7{go to WS7},{groupMoveWindows}{groupWS}
    \keyd            *{move window to WS8},8{go to WS8},{groupMoveWindows}{groupWS}
    \keyd           ({move window to WS9},9{go to WS9},{groupMoveWindows}{groupWS}
    \unusedkey           ),0{}
    \unusedkey            \_,{\textendash}{}
    \unusedkey           +{},={}
    \unusedlongfkey (2\base)  {Backspace}
\ifx \shortkeyboard \defined
    \setlength{\X}{\X + 0.5\base - \sep}
    \unusedfkey     {Insert}
    \unusedfkey          {Home}{}
    \unusedfkey          {Page\\Up}{}
\fi

    \setlength{\Y}{\Y - \base - \sep}
    % upper row
    \X=0\unit
    \longfkeydd      (2\base - \unit) {Tab}{cycle focus forwards}{cycle recent WS},{groupSwitchWindows}{groupWS}
    \keyd           Q{move window to S1},q{go to S1},{groupMoveWindows}{groupScreens}
    \keyd           W{move window to S2},w{go to S2},{groupMoveWindows}{groupScreens}
    \keyd           E{move window to S3},e{go to S3},{groupMoveWindows}{groupScreens}
    \keyd           R{move window to S4},r{go to S4},{groupMoveWindows}{groupScreens}
    \keyd          T{banish cursor},t{unfloat window},{white}{groupMoveWindows}
    \keyd           Y{quit XMonad},y{recompile XMonad},{white}{white}
    \keyd           U{},u{deactivate keybindings},{white}{white}
    \keyd          I{rotate windows CW (except focused)},i{rotate windows CW},{groupMoveWindows}{groupMoveWindows}
    \keyd          O{rotate windows CCW (except focused)},o{rotate windows CCW},{groupMoveWindows}{groupMoveWindows}
    \keyd          P{gmrun},p{gmrun},{groupApplications}{groupApplications}
    \unusedkey           \{{},[{}
    \unusedkey           \}{},]{}
    \unusedlongkey  (\base + \unit) {\textbar},{\textbackslash}
\ifx \shortkeyboard \defined
    \setlength{\X}{\X + 0.5\base - \sep}
    \unusedfkey     {Delete}
    \unusedfkey          {End}{}
    \unusedfkey          {Page\\Down}{}
\fi

    \setlength{\Y}{\Y - \base - \sep}
    % home row
    \X=0\unit
    \longfkeydd (2\base) {Esc (remapped)}{ranger}{open terminal},{groupApplications}{groupApplications}
    \keyd          A{Tall layout},a{toggle fullscreen},{groupLayout}{groupLayout}
    \keyd           S{Half-grid layout},s{toggle mirror},{groupLayout}{groupLayout}
    \keyd           D{Accordion layout},d{swap focused window with master},{groupLayout}{groupMoveWindows}
    \keyd           F{move to next empty WS},f{go to next empty WS},{groupMoveWindows}{groupWS}
    \keyd           G{screenshot (selection)},g{screenshot (all)},{groupApplications}{groupApplications}
    \keyd           H{decrease master pane},h{windows easy motion},{groupLayout}{groupSwitchWindows}
    \keyd          J{cycle stack position\\CW},j{cycle focus forwards},{groupMoveWindows}{groupSwitchWindows}
    \keyd          K{cycle stack position\\CCW},k{cycle focus backwards},{groupMoveWindows}{groupSwitchWindows}
    \keyd           L{increase master pane},l{find open\\ window\\with dmenu},{groupLayout}{groupSwitchWindows}
    \unusedkey           :{},;{}
    \unusedkey           "{},'{}
    \longfkeydd      (2\base + \sep) {Enter}{open terminal}{swap focused window\\with master window},{groupApplications}{groupMoveWindows}

    \setlength{\Y}{\Y - \base - \sep}
    % lower row
    \X=0\unit
    \longfkeyd      (2\base + 2\sep) {Shift}{}{white}
    \keyd          Z{suspend},z{fewer win-\\dows in\\master pane},{groupApplications}{groupLayout}
    \keyd          X{windows easy motion},x{more win-\\dows in\\master pane},{groupSwitchWindows}{groupLayout}
    \keyd           C{kill current window},c{open com-\\munication-related\\ applications},{groupApplications}{groupApplications}
    \keyd          V{toggle\\struts},v{open\\firefox},{groupLayout}{groupApplications}
    \keyd           B{delete link},b{rename workspace},{groupWS}{groupWS}
    \keyd           N{add to link},n{view link},{groupWS}{groupWS}
    \keyd          M{refresh},m{focus master window},{white}{groupSwitchWindows}
    \keyd           {\textless}{},,{more windows in master pane},{white}{groupLayout}
    \keyd           {\textgreater},.{fewer windows in master pane},{white}{groupLayout}
    \unusedkey           ?{},/{}
    \longfkeyd      (3\base) {Shift}{}{white}
\ifx \shortkeyboard \defined
    \setlength{\X}{\X + 0.5\base - \sep + \base + \sep}
    \fkeyd          {$\uparrow$}{\shortcut{k}}
\fi

    \setlength{\Y}{\Y - \base - \sep}
    % space row
    \X=0\unit
    \unusedlongfkey      (1.5\base + \sep) {Ctrl}{}
    \unusedlongfkey (\base)           {\faLinux}
    \longfkeydmod (\base + \sep)    {Alt}{\textbf{Modifier key}\\All bindings as-\\sume this key\\being pressed}
    \longfkeyd      (7\base)          {Space}{gmrun}{groupApplications}
    \longfkeyd (\base + \sep)    {Compose}{}{white}
    \unusedlongfkey (\base)           {\faLinux}
    \unusedlongfkey (\base + 2\sep)   {Menu}
    \unusedlongfkey      (1.5\base + \sep) {Ctrl}{}
\ifx \shortkeyboard \defined
    \setlength{\X}{\X + 0.5\base - \sep}
    \fkeyd          {$\leftarrow$}{\shortcut{h}}
    \fkeyd          {$\downarrow$}{\shortcut{j}}
    \fkeyd          {$\leftarrow$}{\shortcut{l}}
\fi

\end{tikzpicture}


\def\blockA{
    \makeblock{10cm}{Quickstart}{
      To control the window manager keybindings are used. All keybindings consist of the modifier key (Alt-L, called Mod) and some other key(s). The windows are automatically cropped and placed on the screen.
      \begin{itemize}
        \item To start an application, press Mod+Space and enter the application name in the dialog that shows. Use Tab inside the dialog to auto-complete and Enter to launch.
        \item To start firefox, you can use the shortcut Mod+v
        \item Move windows around the different screens by pressing Mod+` (this is the key above Tab)
        \item To see one window in full screen and hide the others behind it, hit Mod+a. To switch back to the tiled view of all windows, hit Mod+a again.
        \item Cycle through the windows by using Mod+j and Mod+k. Alternatively you can use Mod+Shift+Tab, which does the same thingas Mod+j.
      \end{itemize}
      
      For some more information, read the next section.
    }
}
\def\blockB{
    \makeblock{13cm}{Short manual}{
    \vspace{-0.7cm}
    \begin{multicols}{2}
      XMonad is a tiling window manager. This means you usually don't resize windows and move them around, this is done by XMonad for you. Using a set of keybindings, you instead direct XMonad how you want things organized. This principle is often referred to as "tiling" behaviour of a window manager like XMonad.
      
      To start applications, a launcher (called "gmrun") and several shortcuts are configured. Look for them in the color \legendbox{groupApplications}.
      
      XMonad organizes windows in so-called workspaces (WS), similar entities are often called "desktops" elsewhere.
      Every physical screen (S) shows one workspace at a time.
      
      In this configuration, there are 9 numbered workspaces and two special workspaces (called c for communication and s for system).
      There are keybindings for navigating around and reordering the things shown. First of all, one can cycle through windows on the currently focused workspace with the keys highlighted in \legendbox{groupSwitchWindows}. The currently focused window is always highlighted by a thin white border, which is hard to see in some cases.
      
      Moving windows around, for example to different workspaces (WS) or screens (S), is done with keys highlighted in \legendbox{groupMoveWindows}. Reordering the windows inside of a current workspace is done with J, K, I, O, i, o (all with the modifier key, of course). Capital letters denote pressing Mod+Shift+[Letter] instead of Mod+[Letter].
      
      Once several workspaces and screens are populated, you can move around them with the keys highlighted in \legendbox{groupWS} for switching workspaces and \legendbox{groupScreens} for switching screens.
      
      The different ways XMonad places the windows in are summarized by so-called layouts. Switching between different layouts and modifying options of the currently active layout is done by pressing keys in the \legendbox{groupLayout} group.    
      \end{multicols}
    }
}
\def\blockC{
    \makeblock{7cm}{Legend of colors}{
    \vspace{-0.2cm}
      \begin{itemize}
        \item \legendbox{groupApplications} Run commands and applications
        \item  \legendbox{groupSwitchWindows} Select the currently focused window
        \item \legendbox{groupMoveWindows} Move windows around
        \item \legendbox{groupWS} Go to workspace
        \item \legendbox{groupScreens} Go to screen
        \item  \legendbox{groupLayout} Switch or modify the current layout
      \end{itemize}
    }
}

\vspace{+2ex}
\hspace{2ex}
\scriptsize
\Shortunderstack{ {\protect\blockC} }
\Shortunderstack{ {\protect\blockA}  }
\hspace*{4ex}
\Shortunderstack{ {\protect\blockB}  }
%\Shortunderstack{ {\protect\blockK} }
%\Shortunderstack{ {\protect\blockG} {\protect\blockF}
%                   {
%                     \begin{tabular}{r}
%                         \vspace{72pt}\\
%                         \scriptsize \s{for Vifm v0.11 (\today)}\\
%                     \end{tabular}
%                   } }

\end{document}
